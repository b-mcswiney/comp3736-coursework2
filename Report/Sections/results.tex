\section{Results}
\subsection*{Correctness}
\begin{figure}[H]
    \centering
    \includegraphics*[width=0.9\textwidth]{correct_answers_bar_chart.png}
    \caption{Graph showing correctness for all experiements}
\end{figure}
\begin{flushleft}
    From figure 2 we can gather insights on the quality of the questions asked to the participants.
    Within this data two major outliers are apparent, these being question eight and fifteen. The low number of 
    correct answers for these questions indicates that there was a mistake made with them. Upon further investigation,
    this proved to be true as the multiple choice answers contained a typo making them impossible to answer. The responses for questions eight and fifteen 
    are still valid with regards to response time, however, as the participants still answered as if there was no typo.

    In comparison to these outliers, the rest of the data is as we expected with the experiment results. Some questions had a few wrong answers which may indicate they were too difficult
    however, they still had enough correct responses as to indicate there were no mistakes made with the questions.
\end{flushleft}

\subsection*{Time Taken}
\begin{figure}[H]
    \centering
    \includegraphics*[width=0.85\textwidth]{time_taken_box.png}
    \caption{Box plot showing time taken to complete each question}
\end{figure}
\begin{flushleft}
    To start with, we can see from figure 3 that line graphs are consistently better at identifying individual variables in comparison to identifying totals between variables.
    Furthermore, through questions nine to twelve we can identify that area graphs are faster than line graphs at identifying totals of variables. This is most likely because stacked 
    area charts show cumulative values for their variables. On the other hand, we can gather from questions six and eight that area charts are much slower when identifying variables in the middle of
    the stack. This is most likely due to more calculations needing to be performed to find the value.
\end{flushleft}

\subsection*{Feedback from participants}
\begin{figure}[H]
    \centering
    \includegraphics*[width=0.5\textwidth]{user-votes.png}
    \caption{Box plot showing time taken to complete each question}
\end{figure}
\begin{flushleft}
    
\end{flushleft}

% QUESTIONS %
% 01 - Which graph shows more gold medals for GB? - Line
% 02 - Which Graph shows more total medals for USA? - Line
% 03 - Which Graph shows more gold medals for GBe - Area
% 04 - Which Graph shows more total medals for USAe - Area
% 05 - In which graph did GB win less bronze medals? - Line
% 06 - What is the sum of silver medals won by germany in both graphs? - Line
% 07 - In which graph did GB win less bronze medalse - Area
% 08 - What is the sum of silver medals won by germany in both graphse - Area
% 09 - Which tournament shows the least amount of total medals gained? - Line
% 10 - Which tournament shows the most amount of total medals gained? - Line
% 11 - Which tournament shows the least amount of total medals gainede - Area
% 12 - Which tournament shows the most amount of total medals gainede - Area
% 13 - Which tournament shows most gold medals? - Line
% 14 - Which tournament shows most gold medals? - Area
% 15 - Which country got the most medals? - Line
% 16 - Which country got 2nd highest amount of silver medals? - Line
% 17 - Which country got the most medalse - Area
% 18 - Which country got 2nd highest amount of silver medalse - Area
% 19 - Do these graphs have the same dataset?
% 20 - Do these graphs have the same dataset?
% 21 - Which graph type do you prefer?

% GRAPH PREFERANCE %
% LINE - 11111
% AREA - 11111